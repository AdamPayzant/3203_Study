\documentclass{article}

\begin{document}
    \begin{enumerate}
        \item Which of the following physical layer transmission methods doubles the num-ber of bits?
            \newline
            (a) NRZ-I
            \newline
            (b) Manchester
            \newline
            (c) Bipolar
            \newline
            (d) 4B/5B
            \newline
            B because it's the only one that produces double bits
        \item The lowest frequency of a signal is 1,200 $Hz$ and the highest 6,800 $Hz$. What is the bandwidth?
            \newline
            (a) 8,000 $Hz$
            \newline
            (b) 2,800 $Hz$
            \newline
            (c) 5,600 $Hz$
            \newline
            (d) 4,000 $Hz$
            \newline
            C
        \item What is the period of a signal with frequency 0.05?
            \newline
            (a) 20
            \newline
            (b) 40
            \newline
            (c) 10
            \newline
            (d) .95
            \newline
            A: $1/.05 = 20$
        \item A sinusoid wave is given as a function $\omega(t)$ of time $t$ by the formula: $\omega(t) = 6\sin(\frac{\pi}{4} + 40\pi t)$. What is the frequency, phase, and amplitude in this order?
            \newline
            (a) frequency 40, phase $\pi/8$, and amplitude 6
            \newline
            (b) frequency 20, phase $\pi/4$, and amplitude 6
            \newline
            (c) frequency 20, phase $\pi/4$, and amplitude 3
            \newline
            (d) frequency 40, phase $\pi/4$, and amplitude 6
            \newline
            B
        \item What is the redundancy of the 2 dimensional parity check algorithm on $m \times n$ bit words as a function of $m$ and $d$?
            \newline
            (a) $\frac{1}{m} + \frac{1}{n}$
            \newline
            (b) $\frac{1}{mn}$
            \newline
            (c) $1 + \frac{m}{n}$
            \newline
            (d) $1 + \frac{1}{m} + \frac{1}{n}$
            \newline
            D
        \item Why is frequent bit alternation important in physical layer bit streams
            \newline
            (a) To avoid signal collisions
            \newline
            (b) To improve quality of signal
            \newline
            (c) To prevent baseline wander
            \newline
            (d) To prevent bit collisions
            \newline
            C
        \item An adversary is cutting wires (links) to disconnect the nodes of the network depicted. What is the minimum number of links required to be cut so as to partition the network into two connected subnets of size 4 nodes each?
            \newline
            (a) 4
            \newline
            (b) 2
            \newline
            (c) 1
            \newline
            (d) 5
            \newline
            D
        \item The bandwidth of a channel is 200 $Mbps$. What is the transmission delay (in seconds) for a 10 $Kb$ packet?
            \newline
            (a) $a * 10^3 s$
            \newline
            (b) $10^{-1} s$
            \newline
            (c) $\frac{1}{2} * 10^5 s$
            \newline
            (d) $\frac{1}{2} * 10^{-5} s$
            \newline
            D: $\frac{.01}{200} = \frac{1}{2} * 10^{-4} s$
        \item What is the propagation delay on a 6 $km$ long coaxial cable having speed $3 \times 10^8m/s$?
            \newline
            (a) $2 * 10^{-5} s$
            \newline
            (b) $18 * 10^{-11} s$
            \newline
            (c) $\frac{1}{2} * 10^{-5} s$
            \newline
            (d) $18 * 10^{-5} s$
            \newline
            A: $\frac{6}{3x10^8} = 2 * 10^{-5}$
        \item How many bits can a transcontinental channel hold if it has one-way latency of 100 $ms$ and bandwidth of 40 $Mbps$? Express your answer in $Mb$.
            \newline
            (a) 40 $Mb$
            \newline
            (b) 2 $Mb$
            \newline
            (c) 4 $Mb$
            \newline
            (d) 10 $Mb$
            \newline
            C: $.1 * 40$
        \item What's the max number of errors that the $n \times n$ LRC (Longitudinal Redundancy Check) code can detect?
            \newline
            (a) One
            \newline
            (b) Two
            \newline
            (c) $2n$
            \newline
            (d) $\frac{n}{2}$
            \newline
            B
        \item What's the max number of errors that the LRC (Longitudinal Redundancy Check) code can correct?
            \newline
            (a) $n$
            \newline
            (b) $\frac{1}{n}$
            \newline
            (c) Two
            \newline
            (d) One 
            \newline
            D
        \item Give the Hamming distance between the bit strings
            \begin{center}
                0111111011
                \newline
                1110000010
            \end{center}
            (a) 7
            \newline
            (b) 10
            \newline
            (c) 6
            \newline
            (d) 8
            \newline
            C: (Take XOR) 1001111001 = 6
        \item In the CRC code a polynomial is converted to a bit sequence. To what bit sequence is the polynomial $x^7 + x^5 + x^2 + x +1$ converted to?
            \newline
            (a) 10100101
            \newline
            (b) 11100011
            \newline
            (c) 10100111
            \newline
            (d) 10100010
            \newline
            C: $x^7 + 0x^6 + x^5 + 0x^4 + 0x^3 + x^2 + x + 1 = 10100111$
        \item In the selective reject ARQ protocol SREJ
            \newline
            (a) the receiver may reject a packet sent by the sender
            \newline
            (b) both sender and receiver may reject
            \newline
            (c) the sender may reject a packet acknowledged by the receiver
            \newline
            (d) the receiver rejects packet sent twice by the sender
            \newline
            A
        \item In the Stop-and-Wait protocol
            \newline
            (a) the receiver must stop and wait after receiving a packet
            \newline
            (b) the sender must stop and wait after sending a packet
            \newline
            (c) the receiver must stop and wait after receiving a packet in error
            \newline
            (d) the sender must stop and wait after sending a packet in error
            \newline
            B
        \item (a) 12
            \newline
            (b) 14
            \newline
            (c) 13
            \newline
            (d) 15
            \newline
            D
        \item A transmission line has bandwidth 1,200 Hz and signal to noise ration 15. The (Shannon) capacity of the channel is
            \newline
            (a) 2,400
            \newline
            (b) 3,600
            \newline
            (c) 9,600
            \newline
            (d) 4,800
            \newline
            $C = B * log_2(1+R)$ where $R = signal/noise$
            \newline
            $R = 15, B = 1200$
            \newline
            $C = 1200 * log_2(16) = 1200 * 4 = 4800$
            \newline
            D
    \end{enumerate}
\end{document}